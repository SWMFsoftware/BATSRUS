%^CFG COPYRIGHT UM
%\documentclass{book}
%\newcommand{\BATSRUS}{BATS-R-US}
%\begin{document}

\section{IDL Visualization \label{section:idl_visualization}}

This chapter describes the use of the IDL macros for plotting results.
These macros were originally developed for the Versatile Advection Code,
and modified and improved for \BATSRUS. 
All the IDL macros provided for \BATSRUS\ are in the {\bf Idl} directory.

\subsection{IDL path and startup file \label{s-idl-path}}

   It is necessary to let IDL know about the existence of the macros in the 
   Idl directory.  You can define the search path for IDL, for example
\begin{verbatim}
setenv IDL_PATH \
 /home/gtoth/bats/Idl:/usr/local/rsi/idl/lib:/usr/local/rsi/idl/lib/utilities
\end{verbatim}
   for the csh or tcsh shell. Modify the above setting for
   your \BATSRUS\ and IDL libraries as needed. You can also make IDL to read 
   the {\bf Idl/idlrc} file automatically upon start up with
\begin{verbatim}
setenv IDL_STARTUP idlrc
\end{verbatim}
   These environment settings can be put into your \~{}/.login or \~{}/.cshrc 
   files. The above is valid for the csh and tcsh UNIX shells. 

   For other UNIX shells (bash, ksh), use
\begin{verbatim}
IDL_PATH=\
 /home/gtoth/bats/Idl:/usr/local/rsi/idl/lib:/usr/local/rsi/idl/lib/utilities
export IDL_PATH
IDL_STARTUP=idlrc
export IDL_STARTUP
\end{verbatim}
   in the \~{}/.profile file.

\subsection{Running IDL \label{s-run-idl}}

   If the IDL\_PATH and the IDL\_STARTUP variables are set, simply 
   start IDL from the directory where the {\bf *.out} IDL plot files
   and the {\bf *.log} logfiles are, e.g.
\begin{verbatim}
cd run/IO2
idl
\end{verbatim}
   If IDL\_STARTUP is not set, type
\begin{verbatim}
@idlrc
\end{verbatim}
   at the IDL$>$ prompt, so that the commands in the {\bf idlrc} file are
   executed: 
   the procedures in {\bf Idl/procedures.pro, Idl/funcdef.pro} and 
   {\bf Idl/vector.pro} are compiled and 
   the script {\bf Idl/defaults} is executed to set some
   global variables to their default values. You can customize the startup
   of IDL by editing {\bf idlrc} and {\bf Idl/defaults}, e.g. you can
   compile your own IDL subroutines. These subroutines could be put into an 
   {\bf Idl\_mine/} directory which should be then listed in the 
   IDL\_PATH search path.

   If you get trapped by an error inside some IDL routine,
\begin{verbatim}
retall
\end{verbatim}
   will return to the main level. To exit IDL type
\begin{verbatim}
exit
\end{verbatim}
   The alternative 'quit' command is also defined for convenience.

\subsection{Reading a snapshot with getpict \label{s-getpict}}

   To read a single frame from a file, type at the "IDL$>$" prompt of IDL
\begin{verbatim}
.r getpict
\end{verbatim}
   The procedure will prompt you for the {\bf filename}, 
   and it determines the {\bf filetype} and {\bf npictinfiles}
   (the number of snapshots in the file) automatically. Then it asks for
   the frame-number {\bf npict} (1, 2,... npictinfiles) 
   of the snapshot to be read from the file. When {\bf npictinfiles=1},
   the frame number is set to 1 automatically:
\begin{verbatim}
filename(s)   ? example1.out
filetype(s)   = binary
npictinfile(s)=      1
npict=       1
\end{verbatim}
   The header of the file is read and echoed on the screen. The
   type of equation is given {\bf physics} (e.g. 'mhd23') 
   which is normally read from the headline of the file {\bf ExampleA\_mhd23}.
   A typical result:
\begin{verbatim}
headline  =ExampleA
ndim      = 2, neqpar= 1, nw= 8
nx        =    50   50
eqpar     =       1.666667
variables = x y rho ux uy uz p bx by bz gamma
physics (eg. mhd12)=mhd23
it      =           0, time=       0.0000000
Read x and w
GRID            INT       = Array(50, 50)
\end{verbatim}
   At the end, the {\bf x} and {\bf w} variables (containing the coordinates 
   and the plot variables respectively) are read from the file. 
   Note that IDL, unlike FORTRAN, starts indexing from 0 instead of 1. 
   The {\bf GRID} index array is useful for defining
   cuts of the computational domain for plotting, see details 
   in section~\ref{s-plot-part}. 

   If your file contained data on a structured grid, you get the "IDL$>$"
   prompt back, and you can do whatever you want with {\bf x}, {\bf w}, 
   and all the other variables 
   {\bf headline, it, time, gencoord, ndim, neqpar, nw, nx, eqpar, variables}
   defined by the header. 

   You can use the {\bf {\bf plotfunc}} script to get some 
   sophisticated plots or you can use any of the IDL procedures directly, e.g.
\begin{verbatim}
print,time,it
print,nx
print,variables
plot,w(*,10,0)
surface,w1(*,*,2)-w0(*,*,2)
contour,w(*,*,1)
vector,w(*,*,1),w(*,*,2)
plot,x(*,*,0),x(*,*,1),psym=1
\end{verbatim}
   The last command will show the grid points. Another way to plot the
   grid points is to use the {\bf plotgrid} procedures as
\begin{verbatim}
plotgrid,x
\end{verbatim}
   For structured 2D grid, you can plot the grid lines connecting
   the grid points with 
\begin{verbatim}
plotgrid,x,/lines
\end{verbatim}

\subsection{Transformation of non-regular grid \label{s-transform}}

   The file may contain data on a generalized or unstructured 2D grid.
   This is signaled by a negative {\bf ndim} in the plotfile 
   and by the variable {\bf gencoord=1} in IDL.
   A {\bf generalized grid} has the same topology as a regular grid
   but the coordinates are not Cartesian. It is a {it continuouus}
   distortion of the original grid. On the other hand, 
   {\bf an unstructured grid} has the grid points in an arbitrary
   order, therefore the second and third elements of the 
   {\bf nx} array are 1. The AMR grids of \BATSRUS\ and AMRVAC are
   unstructured, while VAC can produce generalized grids.

   In either case the {\bf getpict} macro will ask if you want a
   transformation of the {\bf x} and {\bf w} variables.
   The default behaviour is no transformation (just hit the "Enter" key).
   This choice saves the time of transformation, and uses the original grid 
   and variables, but there are only a few plotmodes available for 
   unstructured grids: contour, contfill, contlabel and contbar.
   For a generalized grid there are two more options: vector and stream.

   Alternatively, you can transform the data onto a regular grid, 
   which makes all the plotmodes fully available. 
   To achieve that select 'regular' transformation with
\begin{verbatim}
transform (r=regular/p=polar/u=unpolar/n=none)=none ? r
\end{verbatim}
   Now the code will interpolate the irregular grid onto a 
   regular one using IDL's {\bf triangulate} and {\bf trigrid} procedures.  
   The size of the regular grid is asked now
\begin{verbatim}
Generalized coordinates, dimensions for regular grid
nxreg(0) (use negative sign to limit x)=      100
nxreg(1) (use negative sign to limit y)=      100
\end{verbatim}
   With these settings the original {\bf w} array is interpolated to 
   a $100\times100$ {\bf wreg} array and the
   coordinates for the regular grid {\bf xreg} are also determined. 
   You can plot the first variable, usually density, in {\bf wreg} the 
   same way as before
\begin{verbatim}
surface,wreg(*,*,0)
\end{verbatim}
   It is possible to restrict the transformation to a rectangular
   part of the original 2D data by using a negative sign for nxreg(0) and/or
   nxreg(1), e.g.
\begin{verbatim}
nxreg(0) (use negative sign to limit x)? -100
xreglimits(0) (xmin)? -15.
xreglimits(2) (xmax)? 30.5
nxreg(1) (use negative sign to limit y)? -50
xreglimits(1) (ymin)? -10.
xreglimits(3) (ymax)? 10.
\end{verbatim}
   Now the $100\times 50$ {\bf xreg} array is limited to the range 
   [-15.,30.5] in x, and [-10.,10.] in y, and this is where the {\bf wreg} 
   array is defined. The {\bf xreglimits} array can be set explicitly too,
\begin{verbatim}
xreglimits=[-15, -10, 30.5, 10]
\end{verbatim}
   Note that the order of the elements is xmin, ymin, xmax, ymax.
   To return to the default behaviour, which is plotting the whole 
   computational domain, set
\begin{verbatim}
xreglimits=0
\end{verbatim}
   If the transformation or the transformation parameters are changed,
   the {\bf .getpict} or {\bf .animate} macros, which read data from
   the disk, will calculate {\bf wreg} with the new tranformation settings
   as expected. On the other hand, the {\bf .r plotfunc} macro, 
   which uses the data in the memory
   and which is designed for speed, normally assumes that the transformation 
   has already been done. This corresponds to the
\begin{verbatim}
dotransform='n'
\end{verbatim}
   setting. If the transformation parameters are changed set
\begin{verbatim}
dotransform='y'
\end{verbatim}
   to force {\bf plotfunc} to redo the transformation as necessary.
   After that you can return to dotransform='n' to save the time of
   transformation.

\subsection{Comparison of data \label{s-compare}}

   You can read snapshots from 2 or at most 3 files for purposes of
   comparison. Simply give the filenames separated by spaces:
\begin{verbatim}
.r defaults
.r getpict
filename(s)   ? example1.out example2.out
filetype(s)   = real4 real4
npictinfile(s)=      21     10
npict?  2
\end{verbatim}
   This will read the second snapshots from 'example1.out' and
   'example2.out'. You may also use wild card characters 
\begin{verbatim}
  *  ?  [ ]
\end{verbatim}
    that are recognized by the Unix 'ls' command, e.g.
\begin{verbatim}
IDL> filename='example[12].out'
IDL> filename='example?.out'
IDL> filename='exampl*.out'
\end{verbatim}
   Note that if any wild card is used then the order of the files will 
   be alphabetical.

   After reading the files with getpict, 
   the coordinates and the conservative variables 
   will be put into {\bf x0, x1} and {\bf w0, w1} respectively, however, 
   the header information, which is printed for each file onto the screen, 
   will be overwritten by the last file read, in this case it will belong to
   {\bf example2.out}. The generic {\bf x, w} arrays will also be
   filled by the data read from the last file, and this is what 
   {\bf plotfunc} plots. If the files contained data on non-regular grid,
   and transform='regular' is set, the data will be interpolated into 
   the arrays {\bf wreg0} and {\bf wreg1}. 
   To compare the two data sets run
\begin{verbatim}
IDL> compare,w0,w1
iw max(|w1-w2|)/max(|w1|+|w2|) sum(|w1-w2|)/sum(|w1|+|w2|)
       0     0.018272938   0.00017745799
       1      0.24608387     0.017349624
       2      0.14307581     0.016188008
       3 wsum=0
       4     0.022624079   0.00022667312
       5     0.014965503   0.00022646554
       6     0.018034518   0.00020733169
       7 wsum=0
\end{verbatim}
or add the wnames array (an optional argument) to get
\begin{verbatim}
IDL> compare,w0,w1,wnames
iw max(|w1-w2|)/max(|w1|+|w2|) sum(|w1-w2|)/sum(|w1|+|w2|)
rho     0.018272938   0.00017745799
mx      0.24608387     0.017349624
my      0.14307581     0.016188008
mz wsum=0
e     0.022624079   0.00022667312
bx     0.014965503   0.00022646554
by     0.018034518   0.00020733169
bz wsum=0
\end{verbatim}
   The comparison shows the maximum difference divided by the sum of maximum 
   absolute
   values and the sum of absolute differences divided by the sum of absolute
   values for each variable in {\bf w0} and {\bf w1}. If a variable is zero
   everywhere both in {\bf w0} and {\bf w1}, the {\bf wsum=0} message is shown.
   You can compare arbitrary 1, 2, 3 and 4 dimensional arrays as long as they 
   have the same size.
   The last dimension is interpreted as the variable index {\bf iw}.
   E.g. you could compare two cuts of {\tt wreg} with
\begin{verbatim}
compare,wreg(0:20,2,*),wreg(0:20,3,*)}
\end{verbatim}
   or you can check if the data read from two files have the same grid
\begin{verbatim}
compare,x0,x1
\end{verbatim}
   If the two files have different resolutions, the {\bf coarsen} 
   function can be used. For example, if two solutions were obtained on 
   100 x 50 and 300 x 150 grids, then 
\begin{verbatim}
rholow =w0(*,*,0)
rhohigh=coarsen(w1(*,*,0), 3)
print,total(abs(rholow-rhohigh))/100/50
\end{verbatim}
   will give the average deviation in density. The coarsening is done in
   finite volume sense, i.e. the fine cells within the coarsened cell are
   averaged out. The {\bf coarsen} function works properly for
   uniform Cartesian grids only. Pointwise values can be compared
   with the use of the {\bf triplet} function (see section \ref{s-plot-part}).


\subsection{Plotting data with plotfunc \label{s-plotfunc}}

   Once the data is read by {\bf getpict} or {\bf animate} you can plot 
   functions of {\bf w} with
\begin{verbatim}
.r plotfunc
\end{verbatim}
   You will see the value of {\bf physics} for the last file read and some 
   parameters with standard default values for different plotting routines:
\begin{verbatim}
physics (e.g. mhd12)      =mhd23
======= CURRENT PLOTTING PARAMETERS ================
ax,az=  30, 30, contourlevel= 30, velvector= 200, velspeed (0..5)= 5
multiplot= 0 (default), axistype (coord/cells)=coord, fixaspect=1
bottomline=3, headerline=0
\end{verbatim}
   The plots are normally shown in physical coordinates, i.e.
   {\bf axistype='coord'}, but the axes can also run in cell indices 
   if {\bf axistype='cells'} is set (that works for structured grid only!).
   If {\bf fixaspect=1} the aspect ratio of the plot will be the same
   as the true aspect ratio of the two axes, while {\bf fixaspect=0} lets
   the plot fit into the plotting window tightly.
   The variables {\bf bottomline} and {\bf headerline} control the number
   of values shown at the bottom from {\bf time, it, nx} and at the top 
   from {\bf headline, nx}.
   You can change these values explicitly (e.g. {\bf bottomline=0}), or change 
   their default values in {\bf Idl/defaults}. 
   See sections \ref{s-plotmode} and \ref{s-plot-part} on the other parameters.

   Now, you will be prompted 
   for the name of function(s) and the corresponding plotting mode(s):
\begin{verbatim}
======= PLOTTING PARAMETERS =========================
wnames                     =  rho ux uy uz p bx by bz
func(s) (e.g. rho p m1;m2) ? rho uy
2D scalar: shade/surface/contour/contlabel/contfill/contbar/tv/tvbar
2D polar : polar/polarlabel/polarfill/polarbar
2D vector: stream/stream2/vector/velovect/ovelovect
plotmode(s)                ? surface
plottitle(s) (e.g. B [G];J)=default
autorange(s) (y/n)         =y
GRID            INT       = Array(50, 50)
\end{verbatim}
   The function(s) to be plotted are determined by the {\bf func}
   string parameter, which is a list of function names separated by spaces.
   The number of functions {\bf nfunc} is thus determined by the number of
   function names listed in {\bf func}. 

   For each function you may set the {\bf plotmode} (in 1D there is no
   choice, since only the {\bf plot} routine is useful). 
   If you give fewer plotmode(s) than {\bf nfunc}, the rest of
   the functions will use the last plotmode given, in the above example
   {\bf surface}. This padding rule is used for all the arrays described 
   by strings. See section~\ref{s-plotmode} for more details on plotmodes.

   The {\bf plottitle} parameter is usually set to {\bf default} which
   means that the function name is used for the title, but you can set it
   explicitly, e.g. {\bf plottitle='Density;Momentum'}. Here the 
   separator character is a semicolon, thus the titles may contain spaces.
   No titles are produced if {\bf plottitle=' '} is set.

   For each function you may set the plotting range by hand or let IDL
   calculate the minimum and maximum by itself. This is defined by
   the {\bf autorange} string parameter, which is a list of 'y' and 'n'
   characters, each referring to the respective function. If you set 'n'
   for any of the variables, the {\bf fmin} and {\bf fmax} arrays have
   to be set, e.g.
\begin{verbatim}
fmin=[1. ,-1.]
fmax=[1.1, 1.]
\end{verbatim}
   IDL remembers the previous setting and uses it, unless the number
   of functions are changed. You can always set fmin=0, fmax=0, and let
   IDL prompt you for the values.

\subsection{Function names in string func \label{s-functions}}

   The function names listed in the {\bf func} string can be any of 
   the variable names listed in the string array {\bf wnames}, which is 
   read from the header of the file, or any of the function name strings 
   listed in the {\bf case} statements in the {\bf funcdef} 
   function in {\bf Idl/funcdef.pro} (see section \ref{s-funcdef}), 
   or any expression using the standard variable, 
   coordinate and equation parameter names 
\begin{verbatim}
x y z   rho p ux uy uz uu bx by bz bb   gamma
\end{verbatim}
   where "uu" and "bb" are the velocity and magnetic field squared, 
   respectively. Note that "x ... bb" are arrays, while  "gamma" is
   a scalar. Here uu and bb are the squared velocity and magnetic field.
   For example the maximum Alfv\'en 
   speed could be given as {\bf func='sqrt(bb/rho)'}, but this is already 
   defined in funcdef.pro as 'calfven'. 

   Expressions can also be formed from variables read from the data files
   even if they not occur among the standard variables listed above.
   These variables should be referred to by the corresponding string 
   in wnames enclosed between curly brackets.
   For example the 10 based logarithm of the oxygen density that is named 
   'OpRho' in wnames can be written as
\begin{verbatim}
alog10({OpRho})
\end{verbatim}
   You may combine two function names with the {\bf ;} character representing
   two components of a vector, 
   e.g. {\bf ux;uy} or {\bf bx;bz}, which can either be plotted as a 
   vectorfield by the {\bf velovect} and {\bf vector} procedures, 
   or as streamlines or fieldlines, using the 
   {\bf stream} and {\bf stream2} plotmodes.
   For other plotmodes the absolute value
   $\sqrt{ux^2+uy^2}$ is plotted.
   You can also put a minus sign in front of any function name, which
   will simply multiply the value of the rest of the string by $-$1. 
   For example '-T' plots ($-$1)*temperature.

\subsection{Plotting modes in string plotmode \label{s-plotmode}}

   There are many plotting modes available for 2D data. These can be
   listed in the {\bf plotmode} for each function separated by spaces.
   If the number of plotmodes is less than the number of functions,
   the last plotmode is applied for the rest of the functions.

   For scalar functions the main plotmodes are 
\begin{verbatim}
Plotmode   Parameters     Description
----------------------------------------------------------------------
contour    contourlevel   contourlines
contlabel  contourlevel   contourlines labeled by value
contfill   contourlevel   levels colored by value
contbar    contourlevel   levels colored by value plus colorbar
polar                     polar contour plot
polarlabel contourlevel   polar contour plot with labels
polarfill  contourlevel   polar plot colored by value
polarbar   contourlevel   polar plot colored by value plus colorbar
tv                        grid cells colored by value
tvbar                     grid cells colored by value plus colorbar
surface    ax az          surface mesh, height proportional to value
shade      ax az          shaded surface, height proportional to value
\end{verbatim}
   The parameters {\bf ax} and {\bf az} define the viewing angle,
   while the {\bf contourlevel} parameter determines the number 
   of contourlevels. The 'tv','tvbar','surface' and 'shade' plotmodes
   can be used for Cartesian grids only (or grids transformed to Cartesian).

   For functions with 2 components (e.g. 'bx;bz') the following plotmodes
   are available
\begin{verbatim}
Plotmode  Parameters  Description
-----------------------------------------------------------------
stream    velvector   stream/fieldlines at random/selected points
          velpos      

vector    velvector   arrows at random/selected positions
          velpos
          velspeed
velovect              arrows at every grid point
ovelovect             arrows at every grid point (for overplot)
\end{verbatim}
   The {\bf velvector} parameter determines the number of arrows or
   stream/fieldlines shown. By default the position of arrows/streamlines
   is random. The positions can be fixed with the {\bf velpos} array
   (see section \ref{s-plot-part} for details). 
   During an animation the arrows can move from their initial position
   parallel to the local velocity at a speed proportional to the magnitude
   of the velocity and the {\bf velspeed} parameter. The maximum value
   is the default {\tt velspeed=5}, while {\tt velspeed=0} does not allow
   the arrows to move. 
   The 'velovect' and 'ovelovect' plotmodes 
   can be used for Cartesian grids only.

   For any of the 2D plotmodes, you can add the following strings
\begin{verbatim}
Option  Description
-----------------------------------------------------------------
log     show the 10 based logarithm of the function
grid    show grid points with plus signs
mesh    show the mesh (lines connecting grid points) 
           only for structured grid
body    show the spherical body with radius {\bf rBody} 
           at the origin as a black circle
over    overplot previous function
white   draw vectors or stream lines with white color even 
           for postscript output (only useful for overplotting)
\end{verbatim}
   Note that it makes no sense to overplot the grid for the 
   {\tt surface} plotting mode, on the other hand {\tt plotmode='shademesh'}
   will plot the shaded surface together with the mesh of 'surface'.
   By default {\bf rBody=3} but this can be changed by hand, or by an equation
   parameter named 'rbody' or 'rBody', which is read from the data file.

   Here is an example for some more complex plotmodes:
\begin{verbatim}
plotmode='contbargridlog streamwhiteoverbody'
\end{verbatim} 
   will show the 10 based logarithm of the first scalar quantity with a color bar 
   and a mesh, and overplot the second vector quantity with white streamlines
   and a black body at the origin.

   For any of the colored plotmodes ('shade', 'contfill', 'contbar', 
   'polarfill', 'polarbar', 'tv', and 'tvbar') the colortable can be 
   changed by one of the 
\begin{verbatim}
xloadct
loadct,3
makect,'red'
\end{verbatim}
   commands. The {\tt xloadct} and {\tt loadct} commands are part of IDL, while
   the {\tt makect} procedure is defined in {\bf Idl/procedures.pro}.
   When no argument is given for {\tt loadct} or {\tt makect}, 
   all the available color tables are listed
   and the choice can be made interactively.

   Many characteristics of the plots can be adjusted with system variables.
   Here is a partial list of these:
\begin{verbatim}
system variable     description
-------------------------------------------------------
!p.charsize         overall character size
!x.charsize         X axis character size
!x.ticks            number of X axis tick marks
!x.tickv            positions of X axis tick marks
!x.tickname         strings at X axis tick marks
!x.minor            number of minor tickmarks
!p.psym             symbols instead of lines in 1D plot
!p.symsize          symbol size
!p.thick            thickness of lines in plots
\end{verbatim}
  The Y and Z axis is are affected by the analogous !y. and !z. variables.

\subsection{Plotting part of the domain \label{s-plot-part}}

   It is possible to plot a part of the simulation domain.
   One way that works for both structured and unstructured grids is
   to set the global system variables
\begin{verbatim}
!x.range=[-10.,10.]
!y.range=[-20.,-5.]
\end{verbatim}
   This will work well for 'flat' plotmodes, like 'contour',
   'contfill', etc. For unstructured grids which are not transformed
   to a regular grid, it works for all the available plotting modes.
   To switch back to the default maximum range, use
\begin{verbatim}
!x.range=0
!y.range=0
\end{verbatim}
   When the data is transfored to a regular grids, the domain of transformation
   can be limited by the {\bf xreglimits} array as described in section
   \ref{s-transform}. 

   For structured grids, there is a further option of limiting or coarsening
   the plot domain, and/or reducing the dimensionality of the plot.
   The {\bf cut} index array selects some part of the function(s)
   determined by {\bf func}.
   This is done {\em after} any grid transformation and {\em after} 
   the functions are calculated so that derivatives can be properly taken 
   by the {\bf funcdef} function.
   The {\bf grid} index array is defined to help to construct
   the {\bf cut} array, e.g. if the grid size is 100 times 100:
\begin{verbatim}
cut=grid(*,50:*)
.r plotfunc
\end{verbatim}
   will show the upper half of the domain. To eliminate the edges use
\begin{verbatim}
cut=grid(1:98,1:98)
\end{verbatim}
   A cross section of the domain can be plotted by reducing
   the number of dimensions of {\bf cut} relative to {\bf grid}:
\begin{verbatim}
cut=grid(*,50)
\end{verbatim}
   will produce 1D plots of the cross section along the midline parallel to 
   the first coordinate axis. For a 2D cut of 3D data, use for example
\begin{verbatim}
cut=grid(*,50,*)
\end{verbatim}
   The effect of the {\bf cut} array can be switched off by 
\begin{verbatim}
cut=0
\end{verbatim}
   or by running {\bf .r defaults}.

   The {\bf triplet} function provides an easy way to set the {\bf cut} 
   index array to represent a coarser grid. This is particularly useful
   for the {\bf velovect} plotmode, which tends to draw too many tiny arrows.
   The {\bf triplet} function can have 3, 6, 9, or 12 parameters depending on 
   the number of dimensions, and each triplet describes a subset of the 
   indices in the given direction. The three elements are the 
   minimum, maximum, and stride (like in Fortran 90), e.g. 
\begin{verbatim}
filename='example2.out'
npict=10
.r getpict
func='ux;uy'
plotmode='velovect'
cut=triplet(0,49,2, 33,66,1)
.r plotfunc
\end{verbatim}
   will show every second cell in the {\bf x} direction and the middle third 
   in the {\bf y} direction. Note that the maximum index value should be
   the actual grid size$-$1 except for the last dimension, otherwise the 
   indices will not be correct. This problem can be solved by the use of the 
   {\bf quadruplet} function, which has four parameters per dimension:
   size, minimum, maximum, and stride. To show a coarse 20$*$20 grid 
   from the top left 40$*$40 part of the 50$*$50 grid use
\begin{verbatim}
cut=quadruplet(50,0,39,2, 50,10,49,2)
.r plotfunc
\end{verbatim}
   Another way to show velocity vectors at fixed positions is the use of 
   the "vector" plotmode after setting the number of vectors {\bf velvector} 
   and the array of positions {\bf velpos(velvector,2)} containing the 
   X and Y coordinates for each vector. In principle {\bf velpos} can
   be defined as an arbitrary set of points.
   For a simple coarsening of the original grid points, the triplet and 
   quadruplet functions can be used again:
\begin{verbatim}
cut=grid(0:39,10:49)
velvector=20*13 & velpos=fltarr(velvector,2)
velpos(*,*)=x(quadruplet(50,1,39,2, 50,11,49,3, 2,0,1,1))
.r plotfunc

cut=0
plotmode='vector'
velvector=25*25 & velpos=fltarr(velvector,2)
velpos(*,*)=x(triplet(0,49,2, 0,49,2, 0,1,1))
.r plotfunc
\end{verbatim}
   Note that the last 0,1,1 triplet and 2,0,1,1 quadruplet correspond to
   the second dimension of {\bf velpos} that always runs from 
   0 (X coordinate) to 1 (Y coordinate). This approach is very useful when 
   the velocity vectors are shown together with plots of other functions 
   that should not be coarsened. To use random positions again, 
   set {\bf velpos=0}.
   The {\bf velpos} array can also be used to position streamlines for 
   plotmode='stream' and 'stream2'.


\subsection{Multiplot \label{s-multiplot}}

   The number and arrangement of subplots is automatically set based 
   on the number of files and and the number of functions. 
   The default arrangement be can overriden by setting the
   {\bf multiplot} array. For example, setting 
   {\bf multiplot=[2,3,0]} gives 2 by 3 subplots filled in row-wise.
   If the third element is 1, the subplots are filled in column-wise.
   Setting {\tt multiplot=3} is identical with {\bf multiplot=[3,1,1]},
   while {\tt multiplot=0} gives the default behaviour.

   Functions can be plot on top of each other by setting the {\bf multiplot}
   array such that the number of subplots is smaller than the number of
   functions. Note that a simpler and more general method is to use the 'over'
   option in the plotmode (see below). 
   To show the magnetic field and density in the same plot, try
\begin{verbatim}
func='rho bx;by'
plotmode='contbar stream'
plottitle='rho and B'
multiplot=1
.r plotfunc
\end{verbatim}
   To show velocity vectors and pressure, use
\begin{verbatim}
func='p ux;uy'
plottitle='p and U'
plotmode='contfill ovelovect'
.r plotfunc
\end{verbatim}
   Note how {\bf plottitle} is set to avoid the default titles
   overlap on top of each other. The {\bf multiplot=1} setting
   is equivalent with {\bf multiplot=[1,1,1]}. In the second example
   'ovelovect' is used (instead of 'velovect') for the velocity to get 
   good alignment with the 'contfill' plot. 

   The same effect can also be achived with the 'over' option in the 
   plotmode and without the use of multiplot:
\begin{verbatim}
func='rho bx;by'
plotmode='contbar streamover'
plottitle='rho and B'
.r plotfunc
func='p ux;uy'
plottitle='p and U'
plotmode='contfill ovelovectover'
.r plotfunc
\end{verbatim}
   The number of functions and the number of subplots can be any combination
   you would like. In 1D plots, the line style is varied for the different 
   functions, so the curves can be distinguished.

\subsection{Plotting another snapshot \label{s-plot-another}}

   If you type
\begin{verbatim}
.r getpict
.r plotfunc
\end{verbatim}
   again, the data will be read and plotted again without any questions asked,
   since IDL remembers the previous settings. 

   If you want to read another frame, say the second, from the same file, type
\begin{verbatim}
npict=2
.r getpict
\end{verbatim}
   You can change the {\bf func} and {\bf plotmode} variables the same way:
\begin{verbatim}
func='rho p'
plotmode='contour surface'
.r plotfunc
\end{verbatim}
   Note that we did not need to reread the data.
   Other variables, all listed in {\bf Idl/defaults}, can be set similarly.
   If you set
\begin{verbatim}
doask=1
\end{verbatim}
   the macros will ask for all the parameters to be confirmed by a simple
   RETURN, or to be changed by typing a new value. Set {\bf doask=0} to
   get the default behaviour, which is no confirmation asked.
   To overplot previous plots without erasing the screen, set
\begin{verbatim}
noerase=1
\end{verbatim}
   You can return to the default settings for all parameters by typing
\begin{verbatim}
.r defaults
\end{verbatim}

\subsection{Animation and plotting with animate \label{s-animate}}

   This general procedure can plot, save into image file(s), or animate 
   (using IDL's Xinteranimate) different functions of data 
   read from one or more files. If a single snapshot is read, the
   plot is drawn without animation. In essence {\bf animate} combines
   {\bf getpict} and {\bf plotfunc} for any number of files and any number 
   of snapshots.
\begin{verbatim}
.r animate
\end{verbatim}
   will first prompt you for {\bf filename(s)} unless already given. 
   Animating more than one input files in parallel is
   most useful for comparing simulations with the same or very similar physics
   using different methods or grid resolution. It is a good idea to save 
   snapshots at the same {\it physical} time into the data files.
   The functions corresponding to the files will be plot columnwise
   with the leftmost column belonging to the first file.
   The headlines and the grid sizes will be shown in 
   for each file separately above the corresponding columns
   if {\bf headerline=2} is set.

   The function(s) to be animated and the plotmode(s) for the functions 
   are determined by the same {\bf func, plotmode}, and {\bf plottitle} 
   strings as for {\bf plotfunc}. If any part of the 
   {\bf autorange} string is set to {\bf 'y'},
   the data file(s) will be read twice: first for setting the common range(s) 
   for all the snapshots and the second time for plotting.
   If {\bf autorange='n'} the file(s) will only be read once.
   The number of snapshots to be animated is limited by the end of 
   file(s) and/or by the {\bf npictmax} parameter. With a formula
\begin{verbatim}
npict=min( npictmax, min( 1 + (npictinfiles-firstpict)/dpict ) )
\end{verbatim}
   The animation runs from {\bf firstpict}, every {\bf dpict}-th picture is
   plotted and the total number of animated frames is at most {\bf npictmax}. 
   If {\bf firstpict} and {\bf dpict} are scalars, the same values are
   applied for all the files, but it is also possible to use different
   values for each file by setting array values, e.g. for two files
\begin{verbatim}
firstpict=[5,9]
dpict    =[1,2]
\end{verbatim}
   will plot every frame starting from the 5th in the first file,
   and every second frame from the 9th in the second file.

   The {\bf multiplot} array can be used to get some really interesting
   effects in {\bf animate}. Besides overplotting different functions,
   as explained in section~\ref{s-multiplot}, 
   the data of different files can also
   be overplotted for comparison purposes.
   Probably it is a good idea to compare 1D slices rather than full 2D plots,
   e.g. 
\begin{verbatim}
filename='example[12].out'
func='rho ux'
cut=grid(*,25)
multiplot=2
.r animate
\end{verbatim}
   will overplot density and velocity read from the two files. 
   The lines belonging to the two data files are distinguished by
   the different line styles. Overplotting two data sets is
   especially useful when the two results are supposed to be identical.

   Timeseries can also be produced easily with {\bf multiplot}. 
\begin{verbatim}
filename='example2.out'
func='rho ux;uy'
plotmode='contfill vectorover'
npictmax=6
multiplot=[3,2,0]
bottomline=1
.r animate
\end{verbatim}
   will show the first 6 snapshots of density with overplot velocity vectors
   in a single plot. 
   Now the time is shown for each plot individually, and setting
   {\bf bottomline=1} limits the time stamp to the most essential
   information, time. An alternative approach is to set the 
   {\bf timetitle} string to format the time and show it as the plot title.
   For example
\begin{verbatim}
timetitle='("t=",f8.1,"s")'
\end{verbatim}
   will show the time as {\bf t=240000.0s}. The time units and the
   initial time (offset) can be set with {\bf timetitleunit} 
   and {\bf timetitlestart}. For example
\begin{verbatim}
timetitleunit=3600.0
timetitlestart=60.0
timetitle='(``t='',f5.2,''h'')'
\end{verbatim}
   will show the time as {\bf t= 6.67h}. Note that timetitleunit is
   relative to the time units used in the data file (e.g. seconds)
   while the offset is given in the time units defined by timetitleunit.

   Set timetitle to an empty string and/or timetitleunit and timetitlestart
   to zero to return to the default behaviour. 
   If {\bf npict*nfile*nplot} is greater than the number 
   of subplots defined by {\bf multiplot}, an animation is done. 
   Type 
\begin{verbatim}
multiplot=0
\end{verbatim}
   to return to default behavior, which is one snapshot per plot.

   Even after exiting from Xinteranimate, the animation can be repeated
   again without rereading the data file(s) by typing
\begin{verbatim}
xinteranimate,/keep_pixmaps
\end{verbatim}
   Sometimes it is interesting to visualize the difference of two runs, e.g.
   to visualize deviations from the initial state, or from a steady state.
   This can be achieved by setting the {\bf wsubtract} array, which will be 
   subtracted from {\bf w} for each snapshot. Note that the subtraction is
   done for the original variables in {\bf w}, 
   so derived quantities should not be plotted. 
   Set {\bf wsubtract=0} to switch off the subtraction.
   Note that this feature is only used by ".r animate", since a single 
   snapshot can be easily manipulated explicitly, e.g. {\bf w=w1-w0}.

\subsection{Slicing structured 3D data \label{s-slice}}

   For visualizing 3D data, .r plotfunc or .r animate can be used after a
   1 or 2D {\bf cut} array has been defined. Alternatively slices of a single
   snapshot (read by .r getpict) can be animated by
\begin{verbatim}
.r slice
\end{verbatim}
   The 3D data is cut along {\bf slicedir}, e.g. for cuts parallel to the
   X-Y plane, set
\begin{verbatim}
slicedir (1, 2, or 3)? 3
\end{verbatim}
   If the grid size is e.g. 50$*$100$*$60, then there are 60 slices to plot.
   The number of animated slices can be reduced:
   at most {\bf nslicemax} slices are shown starting from {\bf firstslice},
   and only every {\bf dslice}-th slice is shown.
   The plots can be further reduced by setting the {\bf cut} array,
   however, now indices in cut refer to a single slice. The {\bf grid2d} 
   index array (generated by the first .r slice, in this case it is a 50$*$100 
   array) can be used, e.g.
\begin{verbatim}
cut=grid2d(*,30:70)
\end{verbatim}
   For {\bf plotmode='vector'} the arrows are not advected with the flow 
   (i.e. velspeed=0) since it does not make sense for the slices.
   If you wish to define the {\bf velpos} array, then the 
   {\bf x2d} array should be used instead of {\bf x}, e.g.:
\begin{verbatim}
velvector=25*50 & velpos=fltarr(velvector,2)
velpos(*,*)=x2d(triplet(0,49,2, 0,99,1, 0,1,1))
\end{verbatim}

\subsection{Function definitions in funcdef \label{s-funcdef}}

   Any function of the variables in {\bf w}, the coordinates in
   {\bf x}, and the equation parameters in {\bf eqpar} can be defined in the 
   {\bf Idl/funcdef.pro} file. Extra information is provided by the equation 
   type {\bf physics} and the variable names {\bf wnames}. The function is 
   identified by the string {\bf f}. The user can easily define new functions
   for a specific application by adding a new {\bf case} statement.
   The modified file should be recompiled before being used:
\begin{verbatim}
.r funcdef
func='mynewfunc'
.r plotfunc
\end{verbatim}
   Some frequently used functions are defined in a rather general way:
   they work for any number of dimensions. Note that some of the functions
   require that the data is saved in consistent units, e.g. 
   pressure and magnetic field squared or velocity and magnetic field
   divided by square root of density have the same units.
\begin{verbatim}
Function name   Meaning
------------------------------------------------------------
pbeta           plasma beta: 2*p_thermal/B^2
T               temperature: p/rho
s               entropy: p_thermal/rho^gamma
csound          sound speed: sqrt(gamma*p_thermal/rho)
cslowx          slow magnetosonic speed along x dimension
cslowy          slow magnetosonic speed along y dimension
cslowz          slow magnetosonic speed along z dimension
calfvenx        Alfven speed along x dimension: bx/sqrt(rho)
calfveny        Alfven speed along y dimension: by/sqrt(rho)
calfvenz        Alfven speed along z dimension: bz/sqrt(rho)
calfven         maximum of Alfven speed: |B|/sqrt(rho)
cfastx          fast magnetosonic speed along x dimension
cfasty          fast magnetosonic speed along y dimension
cfastz          fast magnetosonic speed along z dimension
cfast           maximum of fast speed: sqrt(csound^2+calfven^2)
mach            Mach number: |u|/csound
machx           Mach number:  ux/csound
machy           Mach number:  uy/csound
machz           Mach number:  uz/csound
Mslowx          slow Mach number along x dimension: ux/cslowx
Mslowy          slow Mach number along y dimension: uy/cslowy
Mslowz          slow Mach number along z dimension: uz/cslowz
Malfvenx        Alfven Mach number along x dim: ux/calfvenx
Malfveny        Alfven Mach number along y dim: uy/calfveny
Malfvenz        Alfven Mach number along z dim: uz/calfvenz
Malfven         maximum Alfven Mach number: |u|/calfven
Mfastx          fast Mach number along x dimension: ux/cfastx
Mfasty          fast Mach number along y dimension: uy/cfasty
Mfastz          fast Mach number along z dimension: uz/cfastz
Mfast           maximum fast Mach number: |u|/cfast
\end{verbatim}

\subsection{Reading the logfile with getlog \label{s-getlog}}

One or more (at most three) logfiles can be read by
\begin{verbatim}
.r getlog
\end{verbatim}
   which reads data from the file(s) determined by the {\bf logfilename} 
   parameter. This can be space separated list of file names and/or it
   may include wild card characters. The data in the
   logfile(s) is put into the {\bf wlog} ({\bf wlog1, wlog2}) real arrays,
   while the name of the variables is put into the {\bf wlognames} 
   ({\bf wlognames1, wlognames2}) string arrays. Based on the
   the variable names (e.g. 't' or 'time', 'hour' or 'hours',
   'yr mo dy hr mn sc ms' or 'year month day hour min sec msec') 
   and the associated time data the time in hours is calculated 
   and stored in the {\bf logtime} ({\bf logtime1, logtime2}) 1D real arrays. 
   If no time variables are found, the {\bf logtime} array is set to the
   row index.

   The {\bf wlog(nrow,ncol)} real array contains the rows and columns of the
   logfile, the {\bf wlognames(ncol)} string array the names and the
   {\bf logtime(nrow)} array the time in hour. A simple example is 
\begin{verbatim}
.r getlog
logfilename(s) =log_n020001.log
logfile  =log_n020001.log
headline =Volume averages, fluxes, etc
  wlog(*, 0)= it
  wlog(*, 1)= t
  wlog(*, 2)= dt
  wlog(*, 3)= rho
  wlog(*, 4)= mx
  wlog(*, 5)= my
  wlog(*, 6)= mz
  wlog(*, 7)= p
  wlog(*, 8)= bx
  wlog(*, 9)= by
  wlog(*,10)= bz
  wlog(*,11)= pmin
  wlog(*,12)= pmax
Number of recorded timesteps: nt=      1000
Setting logtime
\end{verbatim}
You can use the IDL plotting procedures directly to visualize the data, e.g.
\begin{verbatim}
plot,logtime,wlog(*,3),xtitle='hour',ytitle='rho_mean'
\end{verbatim}
checks the global mass conservation. 
Use the {\bf plotlog} script to get more sophisticated plots,

\subsection{Plotting the logfile with plotlog \label{s-plotlog}}

Once the data in the logfile(s) have been read with {\bf getlog},
it can be easily visualized with the {\bf plotlog} script.
Simply run
\begin{verbatim}
.r plotlog
logfunc(s)     ? mx pmin pmax
\end{verbatim}
and the script will prompt for the names of the log functions that is
a space separated list of a subset of the strings in the wlognames array. 
To change the list of functions simply change the {\bf logfunc} string.
The time range of the plot can be set with the 2 element
{\bf xrange} array, while the vertical plot range for the individual functions 
can be set by the 2 by nfunc element {\bf yranges} array. For example
\begin{verbatim}
xrange=[10,20]
yranges=[[-10,10],[0,0.1],[0,100]]
\end{verbatim}
The default plot title, the X title (time) and the Y titles (log functions)
can be modified by setting the {\bf title} and {\bf xtitle} strings
and the {\bf ytitles} string array, respectively. For example
\begin{verbatim}
title='Simulation Results'
xtitle='Hours from October 29, 2003'
ytitles=['m!DX!N','P!Dmin!N','P!Dmax!N']
\end{verbatim}
To have no title at all, set these variables to empty strings.
For the default titles, set the variables to 0.

For multiple logfiles the plotlog script will overplot the data.
By default the lines belonging to the different data files are distinguished
by color, but it is possible to use different line styles or symbols
by setting the {\bf linestyles} or {\bf symbols} arrays. For example
\begin{verbatim}
colors=[255,255]
linestyles=[1,2]
\end{verbatim}
will show a dotted and a dashed line with the same default color
(usually white or black). 

Finally the data can be smoothed with the {\bf smooths} array. 
The smoothing width can be defined for each logfile separately. 
A value less than 2 means that the data is not smoothed for that file.
For example
\begin{verbatim}
smooths=[100,0]
\end{verbatim}
will smooth the data from the first file only with a 100 point wide stencil.

\subsection{Saving plots into postscript and graphics files 
                \label{s-postscript}}

In IDL printing a plot is possible through Postscript files.
After the plot looks fine on the screen, use for example
\begin{verbatim}
set_plot,'PS'
device,filename='myfile.ps',xsize=24,ysize=18,/landscape,/color,bits=8
loadct,3
.r plotfunc
device,/close
set_plot,'X'
\end{verbatim}
For a non-color plot omit the {\bf /color,bits=8} parameters and the 
loading of the color table by the {\bf loadct} command. For a {\it portrait}
picture use {\bf xsize=18, ysize=24} and omit the {\bf /landscape} keyword.
If the printout is off the page, set {\bf yoffset=3} too.

An alternative to the above general IDL commands are  
the customized {\bf set\_device} and {\bf close\_device} procedures.
\begin{verbatim}
set_device,'myfile.ps'
loadct,3
.r plotfunc
close_device
\end{verbatim}
The first optional argument of the {\bf set\_device} procedure is the filename.
If it is not given, the default filename 'idl.ps' is used.
There are several keyword arguments too: {\tt /port} for
portrait (default is landscape), {\tt /eps} for encapsulated postscript
file, {\tt percent=80} for an 80\% size reduction, and {\tt psfont=12}
to select a specific font. The {\bf close\_device} procedure
has no arguments, it simply closes the postscript device, and opens the
'X' device.

You can use .r animate instead of .r plotfunc (e.g. for multiple files or 
for time series) in combination with {\tt set\_device} and 
{\tt close\_device}, but make sure that only one plot is produced by setting
{\bf npictmax=1}, and use {\bf firstpict} to select the snapshot.
To save all frames of an animation into a series of Postscript files, 
do not use {\bf set\_device} but set
\begin{verbatim}
savemovie='ps'
\end{verbatim}
This will produce files {\bf Movie/0001.ps,Movie/0002.ps,...} 
in the Movie directory, which should exist. The PostScript files 
are best suited for printing.
You can also save the frames in PNG, TIFF, JPEG or BMP formats, e.g.
by setting 
\begin{verbatim}
savemovie='png'
\end{verbatim}
The frames can be put together into a movie by some program like
{\bf mpeg\_encode} or ImageMagick's {\bf convert}, or 
Apple's {\bf QuicktimePro}. 

\subsection{IDL scripts and procedures \label{s-idl-script}}

All the IDL commands can be collected into a script file, for example 
\begin{verbatim}
Idl_mine/myfig.pro
\end{verbatim}
which can be run from IDL by
\begin{verbatim}
@myfig
\end{verbatim}
This is a convenient way to store the commands for producing complicated 
figures. An example can be found in Idl/EXAMPLE.pro. There are some
restrictions on scripts, however. Loops cannot be used in a script.
If loops are needed, a procedure should be written. From a procedure
one can call all the procedures and functions, but cannot run the
high level scripts. This requires more detailed understanding of the 
IDL visualization procedures.

%\end{document}
